\documentclass{article}
\usepackage{amsmath}
\usepackage{physics}
\usepackage{mathtools}
\usepackage[hidelinks]{hyperref}
\usepackage[giveninits=true]{biblatex}
\usepackage{verbatim}

\addbibresource{{bibliography.bib}}
\def\kb#1#2{| #1 \rangle\!\langle #2 |}
\def\kbtwo#1#2#3#4{|#1 \rangle\!\langle #2 | #3 \rangle\!\langle #4 | }



\title{Phase Estimation?? Hydrogen?? CHEMISTRY?!?!?!}
\author{Floris van den Ende\and Sandy Bridgwater\and Antonio Mendes\and Terts Diepraam}
\date{\today}

\begin{document}

\maketitle

The aim of this project was to implement the algorithm for computing the energy of a hydrogen atom as described by \textcite{poulin}. This algorithm consists of x steps:
\begin{enumerate}
	\item Create three registers: \verb|counting|, \verb|aux| and \verb|main|.
	\item Put the \verb|aux| register in the state $\ket{\beta}$.
	\item Put the \verb|main| register in an eigenstate of the Hamiltonian.
	\item Estimate phase?
	\item ???
	\item Profit?
\end{enumerate}

The state $\ket\beta$ is given by
\[ \ket\beta = B\ket 0 = \sum_j \beta_j \ket j. \]

The code for phase estimation was adapted from~\cite{Qiskit-Textbook}, which is in turn based on~\textcite{nielsen}. The main complication in this part was that the original code was used 3 counting qubits, instead of 2. Therefore, we adapted it to support any number of qubits. Additionally, the bits had to be adapted to fits the registry structure.

Trouble: unitary B not ``normalized'' due to inaccuracies of decimal numbers

\begin{equation}
	\bar H = \frac H{\mathcal{N}} = \sum_j|\beta_j|^2 P_j,
\end{equation}

This is hydrogen Hamiltonian (stolen from~\cite{seeley})

\begin{align*}
\hat{H}_{B K}=&-0.81261 I+0.171201 \sigma_{0}^{z}+0.16862325 \sigma_{1}^{z}-0.2227965 \sigma_{2}^{z}+0.171201 \sigma_{1}^{z} \sigma_{0}^{z} \\
&+0.12054625 \sigma_{2}^{z} \sigma_{0}^{z}+0.17434925 \sigma_{3}^{z} \sigma_{1}^{z}+0.04532175 \sigma_{2}^{x} \sigma_{1}^{z} \sigma_{0}^{x}+0.04532175 \sigma_{2}^{y} \sigma_{1}^{z} \sigma_{0}^{y} \\
&+0.165868 \sigma_{2}^{z} \sigma_{1}^{z} \sigma_{0}^{z}+0.12054625 \sigma_{3}^{z} \sigma_{2}^{z} \sigma_{0}^{z}-0.2227965 \sigma_{3}^{z} \sigma_{2}^{z} \sigma_{1}^{z} \\
&+0.04532175 \sigma_{3}^{z} \sigma_{2}^{x} \sigma_{1}^{z} \sigma_{0}^{x}+0.04532175 \sigma_{3}^{z} \sigma_{2}^{y} \sigma_{1}^{z} \sigma_{0}^{y}+0.165868 \sigma_{3}^{z} \sigma_{2}^{z} \sigma_{1}^{z} \sigma_{0}^{z}
\end{align*}

where $\beta_j = \sqrt{\abs{\alpha_j}/\mathcal{N}}$. Due to the nature of the scaling factor $\mathcal{N}$, $\sum_j \abs{\beta_j}^2 = 1.$
\\\textcite{poulin} define $\beta, B, S$ and  $V$ as following:


\[ S = (B (I - 2 \ket{0}\bra{0}) B^{\dagger}) \otimes I= (I-2\kb \beta \beta) \otimes I) \]
\[ V=\sum_{j}\ket{j}\bra{j} \otimes P_{j} \]



\printbibliography
\end{document}
